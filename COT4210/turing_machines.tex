% chktex-file 13

\documentclass{article}
\usepackage{amsmath}
\usepackage{amssymb}
\usepackage[a4paper, top=0.75in, bottom=0.75in]{geometry}

\title{Turing Machines}
\author{David Robinson}
\date{}
\setlength{\parindent}{0pt}

\begin{document}

\maketitle

\section*{Turing Machine}

A \textbf{Turing Machine} is like a finite automaton with an infinite tape that is used as memory. At every step, a Turing Machine can:
\begin{itemize}
    \item Transition based on the current state and the tape symbol at the current position, like a DFA
    \item Write a symbol to the tape at the current position if it wishes
    \item Move its head either left or right on the tape
\end{itemize}

A Turing Machine is defined by $M=(Q,\Sigma,\Gamma,\delta,q_0, q_\text{accept}, q_\text{reject})$ where:
\begin{itemize}
    \item $Q$ is a finite set of states
    \item $\Sigma$ is the input alphabet, not containing the blank symbol $\square$
    \item $\Gamma$ is the tape alphabet, a superset of $\Sigma$ and contains $\square$
    \item $q_0$ is the start state
    \item $q_\text{accept}$ is the accept state
    \item $q_\text{reject}$ is the reject state, where $q_\text{accept}\neq q_\text{reject}$
\end{itemize}

An \textbf{enumerator} is a Turing machine with a printer. Instead of accepting input, it generates all strings accepted by its language.
\vspace{1em}

A \textbf{multitape Turing machine} has a finite $k$ number of tapes, each with its own read-write head.
\[\delta: Q\times\Gamma^k\rightarrow Q\times\Gamma^k\times{\{L, R\}}^k\]
\vspace{1em}

A \textbf{nondeterministic Turing machine} has a finite number of nondeterministic choices.
\[\delta: Q\times\Gamma\rightarrow\mathbf{P}(Q\times\Gamma\times\{L,R\})\]

\end{document}