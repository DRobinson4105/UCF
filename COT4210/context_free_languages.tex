% chktex-file 13

\documentclass{article}
\usepackage{amsmath}
\usepackage[a4paper, top=0.75in, bottom=0.75in]{geometry}

\title{Context Free Languages}
\author{David Robinson}
\date{}
\setlength{\parindent}{0pt}

\begin{document}

\maketitle

\section*{Context Free Grammar (CFG)}

A \textbf{context-free grammar} is a serios of subsitution rules or productions, where each rule is a transition from one variable to any combination of variables and terminal symbols. The start symbol can be $S$ or the left-hand side of the topmost rule if it is not explcit.
\vspace{1em}

A context-free grammar is defined by $G=(V,\Sigma, R, S)$ where:
\begin{itemize}
    \item $V$ is a finite set called the variables
    \item $\Sigma$ is a finite set, disjoint from $V$, called the terminals
    \item $R$ is a finite set of rules, each rule allowing a variable to be rewritten as a string of variables and terminals
    \item $S\in V$ is the start variable
\end{itemize}

The chain to follow to get the string is called a \textbf{derivation}. A \textbf{leftmost derivation} is a derivation where the leftmost remaining variable is the one replaced. A string $w$ is derived ambiguously in grammar $G$ if it has two or more different leftmost derivations. A grammar $G$ is \textbf{ambiguous} if it generates some string ambiguously.

\subsection*{Chomsky Normal Form}

A context-free grammar is in \textbf{Chomsky Normal Form} if every rule is in one of the following forms:
\begin{itemize}
    \item $A\rightarrow BC$
    \item $A\rightarrow\mathbf{a}$
    \item $S\rightarrow\varepsilon$
\end{itemize}
where $\mathbf{a}$ is a terminal and $A$, $B$, and $C$ are variables.

\subsection*{Pumping Lemma for Context-Free Languages}
If $A$ is a CFL, there is a pumping length $p$ where, if $s\in A$ and $|s|\geq p$, $s=uvxyz$ so that:
\begin{enumerate}
    \item $uv^i xy^i z\in A$ for all non-negative integers $i$
    \item $|vy|>0$
    \item $|vxy|\leq p$
\end{enumerate}

\section*{Pushdown Automata (PDA)}
A \textbf{pushdown automaton} is an NFA with a stack, where on any transition:
\begin{itemize}
    \item Push if $a, \varepsilon\rightarrow a$
    \item Pop if $a, a\rightarrow\varepsilon$
    \item Both if $a, a\rightarrow b$
    \item Neither if $a, \varepsilon\rightarrow\varepsilon$
\end{itemize}

A pushdown automaton is defined by $M=(Q,\Sigma, \Gamma, \delta, q_0, F)$ where:
\begin{itemize}
    \item $Q$ is the set of states
    \item $\Sigma$ is the input alphabet
    \item $\Gamma$ is the stack alphabet
    \item $\delta: Q\times\Sigma_\varepsilon\times\Gamma_\varepsilon\rightarrow P(Q\times\Gamma_\varepsilon)$ is the transition function
    \item $q_0\in Q$ is the start state
    \item $F\subseteq Q$ is the accept states
\end{itemize}

Most PDAs will push \$ to the bottom of the stack to determine if the stack is empty.

\section*{Deterministic Context Free Language (DCFL)}
A deterministic pushdown automaton (DPDA) is like a PDA, but it only has one legal move in any situation. A \textbf{deterministic context free language} is a language that can be recognized by a DPDA.

\subsection*{Key Points}
\begin{itemize}
    \item PDAs and CFGs are equal in power
    \item Every regular language is also a context-free language since a PDA is just an NFA with a stack.
\end{itemize}

\end{document}