\documentclass{article}
\usepackage{amsmath}
\usepackage[a4paper, top=0.75in, bottom=0.75in]{geometry}
\usepackage{enumitem}

\title{Homework 1}
\author{David Robinson}
\date{}
\setlength{\parindent}{0pt}

\begin{document}

\maketitle

\subsection*{Problem 1: Induction Review}

Base:
Let $n=1$. $\sum_{i=1}^1 i = 1$ and $\frac{1^2 + 1}{2} = \frac{1 + 1}{2} = \frac{2}{2} = 1$,
$LHS=RHS$ and the base case holds.
\vspace{1em}

Induction Hypothesis:
Accept $\sum_{i=1}^k i = \frac{k^2 + k}{2}$
\vspace{1em}

Induction:
Show $\sum_{i=1}^{k+1} i = \frac{{(k+1)}^2 + k + 1}{2}$

\[\begin{aligned}
    \sum_{i=1}^{k+1} i &= \sum_{i=1}^k i + k + 1 \quad & \text{summation} \\
    &= \frac{k^2 + k}{2} + k + 1 \quad & \text{induction hypothesis} \\
    &= \frac{k^2 + k}{2} + \frac{2 (k+1)}{2} \quad & \text{identity} \\
    &= \frac{k^2 + k}{2} + \frac{2k + 2}{2} \quad & \text{distribution} \\
    &= \frac{k^2 + k + 2k + 2}{2} \quad & \text{distribution} \\
    &= \frac{k^2 + 2k + 1 + k + 1}{2} \quad & \text{arithmetic} \\
    &= \frac{{(k+1)}^2 + k+1}{2} \quad & \text{arithmetic}
\end{aligned}\]

As desired, and we have shown the conclusion is true for all positive integers.

\subsection*{Problem 2: False Proofs Review}

When $k=1$, the induction hypothesis states that all horses are the same color.
\vspace{1em}

For $k+1=2$, we consider a set of two horses.
\begin{itemize}
    \item Removing the first horse leaves only the second horse in the set and by the induction hypothesis, all horses in the set are the same color.
    \item Removing the second horse leaves only the first horse in the set and by the induction hypothesis, all horses in the set are the same color.
\end{itemize}

While each subset satisfies the induction hypothesis, there is no guarantee that the first and second horse are the same color.

\subsection*{Problem 3: Graph Thinking}

In a graph with $n$ nodes and no self-loops, the degree of each node can range from $0$ (has no edges) to $n-1$ (is connected to every other node).
\vspace{1em}

However, a graph cannot have a node with a degree of $0$ and another with a degree of $n-1$, because if a node is connected to every other node, then every other node must have at least one edge, making it impossible for a node to have a degree of $0$. Therefore, while there are $n$ possible degree values, $[0, n-1]$, the graph can have at most $n-1$ distinct degrees values.
\vspace{1em}

Since there are $n$ nodes but only $n-1$ possible degree values, at least two nodes must have equal degrees.

\subsection*{Problem 4: DFA Construction}

\begin{enumerate}[label=\Alph*.]
    \item First item
\end{enumerate}

\end{document}