\documentclass[8pt,twocolumn]{extarticle}
\usepackage{amsmath}
\usepackage[a4paper, top=0.75in, bottom=0.75in, left=0.75in, right=0.75in]{geometry}

\setlength{\parindent}{0pt}

\begin{document}
Permittivity of free space: $\epsilon_0 = 8.854\times 10^{-12}$

Elementary charge: $e = 1.602\cdot 10^{-19}$

Proton mass: $1.673\cdot 10^{-27}$
Electron mass: $9.11\cdot 10^{-31}$

Speed of light: $c=3\cdot 10^8$
Permeability constant: $\mu_0 = 1.26\times 10^{-6}$
\subsubsection*{Moving Conductor}
\[I = \frac{\Delta V}{R} = \frac{\varepsilon}{R} = \frac{vlB}{R}\]
\[F_\text{mag}=F_\text{pull}=IlB=\frac{vl^2 B^2}{R}\]
\[P_\text{input}=P_\text{dissipated}=I^2 R=\frac{v^2 l^2 B^2}{R}\]
\[\Phi_m = \vec{A}\cdot\vec{B}=|A||B|\cos\theta\quad\text{(uniform magnetic field)}\]

\begin{itemize}
    \item Increasing flux: The induced magnetic field points opposite the applied magnetic
    field.
    \item Decreasing flux: The induced magnetic field points in the same direction as the
    applied magnetic field.
    \item Steady flux: There is no induced magnetic field.
\end{itemize}
\[\varepsilon_\text{induced} = -N\frac{d\Phi_m}{dt}\quad I_\text{induced}=
\frac{\varepsilon_\text{induced}}{R}\]
\[E_\text{inside}=\frac{r}{2}\Big|\frac{dB}{dt}\Big|\quad\text{Solenoid}\]
\subsubsection*{Transformers}
\[\frac{V_2}{V_1} = \frac{N_2}{N_1}\quad P_1=P_2\quad V_1 I_1 = V_2 I_2\quad
\frac{I_1}{I_2}=\frac{N_2}{N_1}\]
\subsubsection*{Inductors}
\[L=\frac{\Phi_m}{I}\text{ henry (H)}\]
\[\Delta V_L = -L\frac{dI}{dt}\quad U_L = L\int_{0}^{I}IdI = \frac{1}{2}LI^2\]
\subsubsection*{LC Circuits}
\[I=-\frac{dQ}{dt}\quad Q(t)=Q_0\cos\omega t\quad \omega = \frac{1}{\sqrt{LC}}\]
\subsubsection*{LR Circuits}
\[I=I_0 e^{-t/(L / R)}\]
\[\tau = \frac{L}{R}\quad\text{where current has decreased to }e^{-1}\]
\subsubsection*{Right-hand rule (wire)}
\begin{enumerate}
    \item Point thumb in the direction of current
    \item Point fingers in the direction of magnetic field
    \item Point palm in the face of force on wire
\end{enumerate}
\[\vec{E}_B = \vec{E}_A + \vec{v}_{BA}\times \vec{B}_A\quad \vec{B}_B = \vec{B}_A - \frac{1}{c^2}
\vec{v}_{BA}\times \vec{E}_A\]
\[a\times b = \begin{vmatrix}
    \hat{\imath} & \hat{\jmath} & \hat{k} \\
    a_x & a_y & a_z \\
    b_x & b_y & b_z
\end{vmatrix}\]
Maxwell changed Ampere's Law because it only applied to steady current.
\[I_\text{disp}=\epsilon_0\frac{d\Phi_e}{dt}=\epsilon_0 \frac{dE\cdot A}{dt}\]
Displacement current is from changing electric field rather than flow of charges.
\[\oint\vec{E}\cdot d\vec{A}=\frac{Q_\text{in}}{\epsilon_0}\quad\text{Gauss's Law}\]
\[\oint\vec{B}\cdot d\vec{A}=0\quad\text{Gauss's Law for Magnetism}\]
\[\oint\vec{E}\cdot d\vec{s} = -N\frac{d\Phi_m}{dt}\quad\text{Faraday's Law}\]
\[\oint\vec{B}\cdot d\vec{s} = \mu_0 I_\text{through} + \epsilon_0 \mu_0 \frac{d\Phi_e}{dt}\quad
\text{Ampere-Maxwell Law}\]
\[\vec{F}=q(\vec{E} + \vec{v} \times \vec{B})\quad\text{Lorentz Force Law}\]
\[E(x, t) = E_0 \cos\left(kx - \omega t + \phi\right)\]
\[c=\frac{\omega}{k}=\frac{\omega\lambda}{2\pi}=f\lambda=\frac{E}{B}\]
\[I=\frac{P}{4\pi r^2}=\frac{c\epsilon_0 E_0^2}{2}\]
\[\langle S \rangle = \frac{1}{\mu_0}EB\sin\theta\quad\text{(Poynting Vector) energy flux of an
electromagnetic wave}\]
\subsubsection*{Right-hand rule (electromagnetic waves)}
Index: electric field, Middle: magnetic field, Thumb: motion
\[I_\text{transmitted}=\frac{1}{2} I_0\quad\text{(unpolarized)}\quad I_\text{transmitted}=I_0\cos^2
\theta\quad\text{(polarized)}\]

\[\varepsilon=\varepsilon_0 \cos\omega t\quad \omega=2\pi f\]
\[v_R = i_R R = V_R\sin\omega t \quad\text{(AC circuit) and $V_R$ is the maximum voltage}\]
\subsubsection*{Capacitor circuit}
\[v_C = V_C\cos\omega t\quad q=Cv_C\]
\[i_C=-\omega CV_C\sin\omega t=\omega CV_C\cos(\omega t + \frac{\pi}{2})\]
\[X_C=\frac{1}{\omega C}\quad I_C=\frac{V_C}{X_C}\quad\text{($X_C$ is Capacitive reactance)}\]
\[\omega_C = \frac{1}{RC}\quad\text{(RC Circuit)}\]
\subsubsection*{Inductor circuit}
An inductor is a coil of wire that generates a magnetic field whne current flows through it and
resist changes in current by inducing an emf opposite to the charge.
\[i_L = I_L\cos(\omega t - \frac{\pi}{2})\]
\[X_L=\omega L\quad I_L=\frac{V_L}{X_L}\quad\text{($X_L$ is Inductive reactance)}\]
\subsubsection*{Series RLC Circuit}
\[V=\sqrt{V_R^2 + {(V_L - V_C)}^2}\]
\[Z=\sqrt{R^2 + {(X_L - X_C)}^2}\quad\text{(impedance)}\quad I_\text{peak}=\frac{\varepsilon_0}{Z}
\]
\[\phi = \tan^{-1}\Big(\frac{X_L - X_C}{R}\Big)\quad\text{(angle between emf and current)}\]
\[\omega_0 = \frac{1}{\sqrt{LC}}\quad\text{(resonance frequency) when $X_L=X_C$ and $Z=R$}\]
\begin{itemize}
    \item If $V_C > V_L$, the circuit operates below resonance frequency
    \item If $V_L > V_C$, the circuit operates above resonance frequency
\end{itemize}
\[p=i\varepsilon\quad\text{(AC circuit) $i$ and $\varepsilon$ are current and potential difference}
\]
\[P_R = \frac{1}{2}I_R^2 R=I_\text{rms}V_\text{rms}\]\
\[x_\text{rms}=\frac{x}{\sqrt{2}}\]
\[P_\text{source}=\frac{1}{2}I\varepsilon_0\cos\phi = I_\text{rms}\varepsilon_\text{rms}\cos\phi=
P_\text{max}\cos^2\phi\]
$\cos\phi$ is the power factor, $\phi$ is the phase between current and emf, $P_\text{mas}=
\frac{1}{2}I_\text{max}\varepsilon_0$.
\begin{itemize}
    \item AC circuit with capacitor: current leads voltage by $\frac{\pi}{2}$ (current reaches
    maximum $\frac{T}{4}$ before voltage)
    \item AC circuit with inductor: current lags voltage by $\frac{\pi}{2}$ (current reaches
    maximum $\frac{T}{4}$ after voltage)
    \item AC circuit with resistor: current is in phase with voltage
\end{itemize}
\end{document}