\documentclass{article}
\usepackage{amsmath}

\begin{document}
\setlength{\parindent}{0pt}
\setlength{\parskip}{1em}
\section*{Electric Field Lines}
Electric field lines are continuous curves which have the same direction as the electric field

\begin{itemize}
    \item Electric field lines are continuous curves tangent to the electric field vectors.
    \item Closely spaced field lines indicate a greater field strength.
    \item Electric field lines start on positive charges and end on negative charges.
    \item Electric field lines never cross.
\end{itemize}

\subsection*{Electric Field Produced by a Continuous Distribution of Charge}
\[\vec{E}(\vec{r})=\int \frac{kdq}{r^2}\hat{r}\]
where $dq$ is a small element of charge at each point in a electric field produced by a continuous
distribution of charge and $r$ is the distance from that small point in the electric field to the
point.

\subsection*{Linear Charge Density}
\begin{itemize}
    \item The linear charge density of an object of length $L$ and charge $Q$ is defined as
    $\lambda=\frac{Q}{L}$
    \item Linear charge density, which has units of $C/m$, is the amount of charge per meter of
    length
\end{itemize}

If the chraged line is infinitely long:
\[E=\frac{2k\lambda}{r}\]

\subsection*{Surface Charge Density}
\begin{itemize}
    \item The surface charge density of a two-dimensional distribution of charge across a surface
    of area $A$ is defined as $\eta = \frac{Q}{A}$.
    \item Surface charge density, which has units of $C/m^2$, is the amount of charge per square
    meter.
\end{itemize}

\section*{Capacitors}

\[E=\frac{V}{d}\]

\begin{itemize}
    \item The electric field inside an ideal parallel plate capacitor is uniform and directed from
    the postiively charged plate to the negatively charged plate.
    \item If you know the surface charge density $\sigma=\frac{Q}{A}$ on the plates, the electric
    field can also be calculated using: $E=\frac{\sigma}{\epsilon_0}$.
    \item Inside the capacitor, the net field points toward the negative plate.
\end{itemize}

\[\vec{E}_{capacitor}=\begin{cases}
\big(\frac{Q}{\epsilon_0 A}\text{, from positive to negative}\big) & \text{inside} \\ 
\vec{0} & \text{outside}
\end{cases}\]

\section*{Electric Dipoles}
Dipole moment $p=q\times d$ where $d$ is a vector directed from negative charge to the positive one
and $q$ is the magnitude of charge.

\[E=\frac{2kp}{r^3}\]

where $\epsilon_0=8.854\times 10^-12\: C^2 / N\cdot m^2$ is the vacuum permittivity

When the dipole moment $\vec{p}$ is at an angle $\theta$ to the field, it causes the dipole to
experience a torque, $\tau = r\times F = p\times E$. Each charge experiences force $F=qE$. The net
force on the
dipole is zero because the forces are of opposite directions so dipole will not move as a whole in
electric field.
\[|\tau | = pE\sin (\theta)\] where $\theta$ is the angle between the dipole moment and the
electric field.

\end{document}