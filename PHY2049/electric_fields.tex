\documentclass{article}
\usepackage{amsmath}
\usepackage{bigints}

\title{Chapter 23 Electric Fields}
\author{David Robinson}
\date{}
\setlength{\parindent}{0pt}

\begin{document}
\maketitle

\section*{Electric Field Models}

\begin{enumerate}
    \item A point charge (small charged objects):
    \[\vec{E}=\frac{1}{4\pi\epsilon_0}\frac{q}{r^2}\hat{r}\]
    \item An infinitely long line of charge (wires):
    \[\vec{E}=\frac{1}{4\pi\epsilon_0}\frac{2|\lambda |}{r}\]
    \item An infinitely wide plane of charge (electrodes):
    \[\vec{E}=\frac{\eta}{2\epsilon_0}\begin{cases}
        \text{away from plane if charge } + \\
        \text{toward plane if charge } -
    \end{cases}\]
    \item A sphere of charge (electrodes):
    \[\vec{E}=\frac{1}{4\pi\epsilon_0}\frac{Q}{r^2}\hat{R}\text{ for }r > R\]
\end{enumerate}

\section*{The Electric Field of Point Charges}

\subsection*{The Electric Field of a Dipole}

Two equal but opposite charges separated by a small distance form an electric dipole. The dipole
moment $\vec{p}=qs$, where $q$ is the positive charge and $s$ is the distance between the charges,
determines the orientation of the dipole and electric field strength. \newline

The electric field on a point on the axis between the two charges:
\[\vec{E}_{\text{dipole}}=-\frac{1}{4\pi\epsilon_0}\frac{2\vec{p}}{r^3} \text{ (on the axis)}\]
where $r$ is the distance measured from the center of the dipole.
\[\vec{E}_{\text{dipole}}=-\frac{1}{4\pi\epsilon_0}\frac{\vec{p}}{r^3} 
\text{ (in the bisecting plane)}\]

\subsection*{Electric Field Lines}

\begin{itemize}
    \item Electric field lines are continuous curves tangent to the electric field vectors.
    \item Closely spaced field lines indicate a greater field strength.
    \item Electric field lines start on positive charges and end on negative charges.
    \item Electric field lines never cross.
\end{itemize}

\section*{The Electric Field of a Continuous Charge Distribution}

\subsection*{Charge Density}
\begin{itemize}
    \item The linear charge density of an object of length $L$ and charge $Q$ is defined as
    $\lambda=\frac{Q}{L}$.
    \[Q=\bigintssss_0^L\lambda dx\]
    \item Linear charge density, which has units of $C/m$, is the amount of charge per meter of
    length.
    \item The surface charge density of a surface with area $A$ and charge $Q$ is defined as
    $\eta=\frac{Q}{A}$.
    \item Surface charge density, which has units of $C/m^2$, is the amount of charge per square
    meter.
    \item The volume charge density of an object with volume $V$ and charge $Q$ is defined as
    $\rho=\frac{Q}{V}$.
    \item Volume charge density, which has units of $C/m^3$, is the amount of charge per cubic
    meter.
\end{itemize}

\section*{The Parallel-Plate Capacitor}

A parallel-plate capacitors is the arrangement of two electrodes (metal plates) face-to-face and
charged equally but oppositely.
\subsection*{Properties}

\[\vec{E}_\text{capacitor}=\begin{cases}
    \Big(\frac{Q}{\epsilon_0 A}\text{, from positive to negative}\Big) & \text{inside} \\
    \vec{0} & \text{outside}
\end{cases}\]
\begin{itemize}
    \item Capacitors are charged by transferring electrons from one plate to the other.
    \item Capacitors have a uniform electric field between the electrodes, meaning that the
    electric field from the electrodes is the same in strength and direction.
\end{itemize}

\section*{Motion of a Dipole in an Electric Field}
When the dipole moment $\vec{p}$ is at an angle $\theta$ to the field, it causes the dipole to
experience a torque, $\tau = r\times F = p\times E$. Each charge experiences force $F=qE$. The net
force on the
dipole is zero because the forces are of opposite directions so dipole will not move as a whole in
electric field.

\[|\tau | = pE\sin \theta\] where $\theta$ is the angle between the dipole moment and the
electric field.


\end{document}