\documentclass{article}
\usepackage{amsmath}
\usepackage{bigints}

\title{Gauss's Law Chapter 24}
\author{David Robinson}
\date{}
\setlength{\parindent}{0pt}

\begin{document}

\maketitle

\section*{Symmetry}
The symmetry of the electric field must match the symmetry of the charge distribution.

\subsection*{Fundamental Symmetries}
\begin{enumerate}
    \item Planar symmetry: The field is perpendicular to the plane.
    \item Cylindrical symmetry: The field is radial toward or away from the axis.
    \item Spherical symmetry: The field is radial toward or away from the center
\end{enumerate}

\section*{Electric Flux}
Electric flux, $\Phi$, is the amount of the electric field $E$ that passes through a surface with area $A$.
Electric field only flows in one direction so $E_\perp=E\cos\theta$ is the only part of $E$ that passes through a surface with angle $\theta$ between the orientation of that surface and the electric field.
\[\Phi = E_\perp A = EA\cos\theta\]

\end{document}