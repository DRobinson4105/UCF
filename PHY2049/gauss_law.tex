\documentclass{article}
\usepackage{amsmath}
\usepackage{bigints}

\title{Gauss's Law Chapter 24}
\author{David Robinson}
\date{}
\setlength{\parindent}{0pt}

\begin{document}

\maketitle

\section*{Symmetry}
The symmetry of the electric field must match the symmetry of the charge distribution.

\subsection*{Fundamental Symmetries}
\begin{enumerate}
    \item Planar symmetry: The field is perpendicular to the plane.
    \item Cylindrical symmetry: The field is radial toward or away from the axis.
    \item Spherical symmetry: The field is radial toward or away from the center
\end{enumerate}

\section*{Electric Flux}
Electric flux, $\Phi$, is the amount of the electric field $E$ that passes through a surface with area $A$.
Electric field only flows in one direction so $E_\perp=E\cos\theta$ is the only part of $E$ that passes through a surface with angle $\theta$ between the orientation of that surface and the electric field.
\[\Phi = E_\perp A = EA\cos\theta\]

\subsection*{Nonuniform Electric Field}
A surface in a nonuniform electric field can be split into small pieces where the electric flux of each area can be represented by $\delta\Phi_i = \vec{E}_i\cdot\delta\vec{A}_i$
The total electric flux through a surface can then be represented by
\[\Phi_e =\bigintssss \vec{E}\cdot d\vec{A}\]

\begin{enumerate}
    \item If the electric field is everywhere tangent to a surface, the electric flux through the surface is $\Phi_e = 0$.
    \item If the electric field is everywhere perpendicular to a surface and has the same magnitude $E$ at every point, the electric flux through the surface is $\Phi_e = EA$.
\end{enumerate}

\subsection*{Calculating Electric Flux through a Closed Surface}
\begin{enumerate}
    \item Choose a Gaussian surface made up of pieces that are everywhere tangent to the electric field or everywhere perpendicular to the electric field.
    \item Use the previous rules to evaluate the surface integrals over these surfaces, then add the results.
\end{enumerate}
\textbf{The net electric flux is zero through a closed surface that does not contain any net charge.}

\section*{Gauss's Law}
The flux through any closed surface surrounding charges that sum to $Q_{in}$ is
\[\Phi_e = \bigintssss \vec{E}\cdot d\vec{A} = \frac{Q_{in}}{\epsilon_0}\]

\subsection*{Calculating Electric Flux in a spherical Gaussian surface inside a uniform sphere of charge}
Where $R$ is the radius of the sphere of charge and $r$ is the radius of the Gaussian surface
\[\Phi_e = E\times 4\pi r^2 = \frac{Q_{in}}{\epsilon_0}\]

Because the charge distribution is uniform, the volume charge density is
\[\rho = \frac{Q}{V_R} = \frac{Q}{\frac{4}{3}\pi R^3}\]

\[Q_{in} = \rho V_r=\frac{Q}{\frac{4}{3}\pi R^3}\times\frac{4}{3}\pi r^3=Q\frac{r^3}{R^3}\]

Using $Q_{in}$ back into Gauss's law results in
\[E\times 4\pi r^2 = \frac{Qr^3 / R^3}{\epsilon_0}\rightarrow E=\frac{Qr}{4\pi\epsilon_0 R^3}\]

\section*{Conductors in Electrostatic Equilibrium}
If a charged conductor is in electrostatic equilibrium, then there is no current through the conductor and the charges are all stationary. The electric field is zero at all points inside a conductor in electrostatic equilibrium. Since the electric field is zero at all points inside the conductor, the electric flux and net charge is also zero.\newline

If there's no net charge in the interior of a conductor in electrostatic equilibrium, then all the excess charge on a charged conductor resides on the surface of the conductor.
\[\vec{E}_{\text{surface}}=\frac{\eta}{\epsilon_0}\text{ and is perpendicular to surface}\]

\subsection*{Screening}
The electric field can be excluded from a region of space in a parallel-plate capacitor by surrounding it with a conducting box. The conducting box will be polarized and induce surface charges and the internal electric field will be $\vec{E}=0$.

\subsection*{Additional Notes}
\begin{enumerate}
    \item The electric flux of a point charge through a spherical surface is independent of the radius of the sphere as $E$ will decrease in proportion to $1/r^2$ but $A$ will increase in proportion to $r^2$.
\end{enumerate}

\end{document}