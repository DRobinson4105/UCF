\documentclass{article}
\usepackage{amsmath}
\usepackage{amssymb}
\usepackage{txfonts}

\title{Chapter 7 Symmetric Matrices and Quadratic Forms}
\author{David Robinson}
\date{}
\setlength{\parindent}{0pt}

\begin{document}
\maketitle

\section*{Diagonalization of Symmetric Matrices}

A \textbf{symmetric} matrix is a matrix $A$ such that $A^T = A$.

\subsubsection*{Theorem 1}
If $A$ is symmetric, then any two eigenvectors from different eigenspaces are orthogonal.

\subsubsection*{Theorem 2}
An $n\times n$ matrix $A$ is orthogonally diagonalizable if and only if $A$ is a symmetric matrix.

\subsubsection*{Theorem 3 --- The Spectral Theorem for Symmetric Matrices}
An $n\times n$ symmetric matrix $A$ has the following properties:
\begin{enumerate}
    \item $A$ has $n$ real eigenvalues, counting multiplicities.
    \item The dimension of the eigenspace for each eigenvalue $\lambda$ equals the multiplicity of
    $\lambda$ as a root of the characteristic equation.
    \item The eigenspaces are mutually orthogonal, in the sense that eigenvectors corresponding to
    different eigenvalues are orthogonal.
    \item $A$ is orthogonally diagonalizable.
\end{enumerate}

\subsubsection*{Spectral Decomposition}
\[A=\lambda_1\mathbf{u}_1\mathbf{u}_1^T+\lambda_2\mathbf{u}_2\mathbf{u}_2^T+\cdots +\lambda_n
\mathbf{u}_n\mathbf{u}_n^T\]

\pagebreak

\subsection*{Key Points}
\begin{enumerate}
    \item A matrix $U$ is orthogonal if $U^T U = I$, and if so, $U^T=U^{-1}$.
    \item A matrix $A$ can be orthogonally diagonalized by finding the $n$ eigenvalues and forming
    $D$ as a diagonal matrix of the eigenvalues and $P$ as the normalized orthogonal eigenvectors
    for the eigenvalues. (Use Gram-Schmidt Process to form orthogonal basis from eigenvectors).
    \item Multiplying a column vector $u$ of $\mathbb{R}^n$ on the right by $u^T x$ is the same as
    multiplying the column vector by the scalar $u\cdot x$.
\end{enumerate}

\pagebreak

\section*{The Singular Value Decomposition}
Let $A$ be an $m\times n$ matrix. The singular values of $A$ are the square roots of the
eigenvalues of $A^T A$, and they are arranged in decreasing order.

\subsubsection*{Theorem 9}

Suppose $\{\mathbf{v}_1, \ldots, \mathbf{v}_n\}$ is an orthonormal basis of $\mathbb{R}^n$
consisting of eigenvectors of $A^T A$, arranged so that the corresponding eigenvalues of $A^T A$
satisfy $\lambda_1 \geq \cdots \geq \lambda_n$, and suppose $A$ has $r$ nonzero singular values.
Then $\{A\mathbf{v}_1, \ldots, A\mathbf{v}_r\}$ is an orthogonal basis for Col $A$, and rank $A=r$.

\subsubsection*{Theorem 10 --- The Singular Value Decomposition}

Let $A$ be an $m\times n$ matrix with rank $r$. Then there exists an $m\times n$ matrix $\Sigma$ as
in
\[\Sigma = \begin{bmatrix}
    D & 0 & \cdots & 0 \\ 0 & 0 & \cdots & 0 \\ \vdots & \vdots & \ddots & 0 \\ 0 & 0 & 0 & 0
\end{bmatrix}\]

for which the diagonal entries in $D$ are the first $r$ singular values of $A$,
$\sigma_1 \geq \sigma_2 \geq \cdots \geq \sigma_r > 0$, and there exist an $m\times m$ orthogonal
matrix $U$ and an $n\times n$ orthogonal matrix $V$ such that
\[A=U\Sigma V^T\]
\begin{enumerate}
    \item $U=\{u_1, \ldots, u_r\}$ where $u_i=\frac{1}{\sigma_i}Av_i$.
    \item $V$ contains the unit eigenvectors for $A^T A$.
\end{enumerate}

\subsubsection*{The Invertible Matrix Theorem (concluded)}
Let $A$ be an $n\times n$ matrix. Then the following statements are each equivalent to the
statement that $A$ is an invertible matrix:

\begin{enumerate}
    \item ${(\text{Col }A)}^\perp = \{\mathbf{0}\}$.
    \item ${(\text{Nul }A)}^\perp = \mathbb{R}^n$.
    \item $\text{Row }A=\mathbb{R}^n$.
    \item $A$ has $n$ nonzero singular values.
\end{enumerate}

\subsection*{Key Points}
\begin{enumerate}
    \item Let $A$ be an $m\times n$ matrix. If there are more columns than rows, $m<n$, then the
    singular value decomposition can be solved by first calculating the singular value
    decomposition of $A^T$. $A^T=U\Sigma V^T\rightarrow {(A^T)}^T={(U\Sigma V^T)}^T=V\Sigma^T U^T$.
    \item $\text{det }A=(\text{det }U)(\text{det }\Sigma)(\text{det }V^T)$.
\end{enumerate}

\end{document}