\documentclass{article}
\usepackage{amsmath}

\begin{document}

\section*{Matrices}

A matrix is in \textbf{echelon form} if:
\begin{enumerate}
    \item All nonzero rows are above any rows of all zeros
    \item Each leading entry of a row is in a column to the right of the leading entry of the row
    above it
    \item All entries in a column below a leading entry are zeros
\end{enumerate}

\noindent
A matrix is in \textbf{reduced echelon form} if:
\begin{enumerate}
    \item It is in echelon form
    \item The leading entry in each nonzero row is 1
\end{enumerate}

\noindent
A linear system is consistent if and only if the rightmost column of the augmented matrix is not a
pivot column. That is, if and only if an echelon form of the augmented matrix has no row of the
form $\begin{bmatrix} 0 & \cdots & 0 & b \end{bmatrix}$ with $b$ being nonzero.

\end{document}