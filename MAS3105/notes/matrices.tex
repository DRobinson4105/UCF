\documentclass{article}
\usepackage{amsmath}
\usepackage{amssymb}

\begin{document}
\setlength{\parindent}{0pt}

\section*{Linear Equations in Linear Algebra}

A matrix is in \textbf{echelon form} if:
\begin{enumerate}
    \item All nonzero rows are above any rows of all zeros
    \item Each leading entry of a row is in a column to the right of the leading entry of the row
    above it
    \item All entries in a column below a leading entry are zeros
\end{enumerate}

\noindent
A matrix is in \textbf{reduced echelon form} if:
\begin{enumerate}
    \item It is in echelon form
    \item The leading entry in each nonzero row is 1
\end{enumerate}

\subsection*{Properties}
\begin{enumerate}
    \item A linear system is consistent if the rightmost column of echelon form of the augmented matrix is not a pivot column.
    \item Two matrices are row equivalent if there exists a sequence of elementary row operations that transforms one matrix into the other.
    \item The two fundamental questions are about whether the solution exists and whether there is only one solution.
\end{enumerate}
\end{document}
