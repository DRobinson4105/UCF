\documentclass{article}
\usepackage{amsmath}
\usepackage{amssymb}
\usepackage[a4paper, top=0.25in, bottom=0.25in, left=0.25in, right=0.25in]{geometry}

\setlength{\parindent}{0pt}

\begin{document}

Multi-variable limit check: Check when $x=0$, $y=0$, and $x=y$ or check if limit depends on $\theta$ when converting to polar coordinates
\vspace{1em}

Ellipsoid: $\frac{x^2}{a^2} + \frac{y^2}{b^2} + \frac{z^2}{c^2} = 1$ (xy-plane: ellipse, xz-plane: ellipse, yz-plane: ellipse)
\vspace{1em}

Hyperboloid of One Sheet: $\frac{x^2}{a^2} + \frac{y^2}{b^2} - \frac{z^2}{c^2} = 1$ (xy-plane: ellipse, xz-plane: hyperbola, yz-plane: hyperbola)
\vspace{1em}

Hyperboloid of Two Sheets: $\frac{x^2}{a^2} - \frac{y^2}{b^2} - \frac{z^2}{c^2} = 1$ (xy-plane: hyperbola, xz-plane: hyperbola, yz-plane: ellipse)
\vspace{1em}

Elliptic Cone: $\frac{x^2}{a^2} + \frac{y^2}{b^2} - \frac{z^2}{c^2} = 0$ (xy-plane: ellipse, xz-plane: two intersecting lines, yz-plane: two intersecting lines)
\vspace{1em}

Elliptic Paraboloid: $\frac{x^2}{a^2} + \frac{y^2}{b^2} = z$ (xy-plane: ellipse, xz-plane: parabola, yz-plane: parabola)
\vspace{1em}

Hyperbolic Paraboloid (Saddle): $\frac{x^2}{a^2} - \frac{y^2}{b^2} = z$ (xy-plane: hyperbola, xz-plane: parabola, yz-plane: negative parabola)
\vspace{1em}

Clairaut's Theorem: Suppose that $f(x, y)$ is defined on an open desk $D$ that contains the point $(a, b)$. If the functions $f_{xy}$ and $f_{yx}$ are continuous on $D$, then $f_{xy}=f_{yx}$.

\[\nabla f(x,y)=f_x(x,y)\:\mathbf{i}+f_y(x,y)\:\mathbf{j}\]
Directional derivative of $f$ in the direction of $u$ is
\[D_u f(a, b)=\lim_{h\rightarrow 0}\frac{f(a+h\cos\theta, b+h\sin\theta)-f(a,b)}{h}\]
\[D_u f(x, y) = \nabla f(x,y)\cdot\mathbf{u} = f_x(x, y)\cos\theta + f_y(x,y)\sin\theta\]
\[T=L(x,y)=f(x_0,y_0)+f_x(x_0,y_0)(x-x_0)+f_y(x_0,y_0)(y-y_0)\]

$D_u f(x_0, y_0)$ is maximized when $u$ points in the same direction as $\nabla f(x_0, y_0)$. The maximum value of $D_u f(x_0, y_0)$ is $\|\nabla f(x_0, y_0)\|$.
\vspace{0em}

$D_u f(x_0, y_0)$ is minimized when $u$ points in the opposite direction from $\nabla f(x_0, y_0)$. The minimum value of $D_u f(x_0, y_0)$ is $-\|\nabla f(x_0, y_0)\|$.
\vspace{1em}

A point is a critical point of a function of two variables if $f_x(x_0,y_0)=f_y(x_0,y_0)=0$ or either $f_x(x_0,y_0)$ or $f_y(x_0,y_0)$ does not exist.
\vspace{1em}

Second Derivative Test: If the first and second-order partial derivatives are continuous,
\[D=f_{xx}(x_0,y_0)f_{yy}(x_0,y_0)-{(f_{xy}(x_0,y_0))}^2\]
\begin{enumerate}
    \item If $D>0$ and $f_{xx}(x_0,y_0)>0$, then $f$ has a local minimum at $(x_0,y_0)$.
    \item If $D>0$ and $f_{xx}(x_0,y_0)<0$, then $f$ has a local maximum at $(x_0,y_0)$.
    \item If $D<0$, then $f$ has a saddle point at $(x_0,y_0)$.
    \item If $D=0$, then the test is inconclusive.
\end{enumerate}

Finding local extema in a bounded region:
\begin{enumerate}
    \item Test extreme points on boundary
    \item Test all boundary lines: Consider a line segment connecting $(a,c)$ and $(b,c)$. It can be parameterized by the equations $x(t)=t$, $y(t)=c$ for $a\leq t\leq b$. Define $g(t)=f(x(t),y(t))$. The critical points are where $g'(t)=0$.
    \item Test whole bounded region
\end{enumerate}
\end{document}