\documentclass{article}
\usepackage{amsmath}
\usepackage{amssymb}
\usepackage[a4paper, top=0.75in, bottom=0.75in]{geometry}

\title{Extrema}
\author{David Robinson}
\date{}
\setlength{\parindent}{0pt}

\begin{document}

\maketitle

\section*{Critical Points}

Let $z=f(x,y)$ be a function of two variables that is defined on an open set containing the point $(x_0,y_0)$. The point $(x_0,y_0)$ is called a \textbf{critical point of a function of two variables} $f$ if one of the two following conditions holds:
\begin{enumerate}
    \item $f_x(x_0,y_0)=f_y(x_0,y_0)=0$
    \item Either $f_x(x_0,y_0)$ or $f_y(x_0,y_0)$ does not exist.
\end{enumerate}

Let $z=f(x,y)$ be a function of two variables that is defined and continuous on an open set containing the point $(x_0,y_0)$. Then $f$ has a \textbf{local maximum} at $(x_0,y_0)$ if
\[f(x_0,y_0)\geq f(x,y)\]
for all points $(x,y)$ within some disk centered at $(x_0,y_0)$. The number $f(x_0,y_0)$ is called a local maximum value. If the preceding inequality holds for every point $(x,y)$ in the domain of $f$, then $f$ has a global maximum at $(x_0,y_0)$.
\vspace{1em}

The function $f$ has a \textbf{local minimum} at $(x_0,y_0)$ if
\[f(x_0,y_0)\leq f(x,y)\]
for all points $(x,y)$ within some disk centered at $(x_0,y_0)$. The number $f(x_0,y_0)$ is called a local minimum value. If the preceding inequality holds for every point $(x,y)$ in the domain of $f$, then $f$ has a global minimum at $(x_0,y_0)$.
\vspace{1em}

If $f(x_0,y_0)$ is either a local maximum or local minimum value, then it is called a local extremum.

\subsection*{Fermat's Theorem for Functions of Two Variables}

Let $z=f(x,y)$ be a function of two variables that is defined and continuous on an open set containing the point $(x_0,y_0)$. Suppose $f_x$ and $f_y$ each exists at $(x_0,y_0)$. If $f$ has a local extremum at $(x_0,y_0)$, then $(x_0,y_0)$ is a critical point of $f$.
\vspace{1em}

Given the function $z=f(x,y)$, the point $(x_0,y_0,f(x_0,y_0))$ is a saddle point if both $f_x(x_0,y_0)=0$ and $f_y(x_0,y_0)=0$, but $f$ does not have a local extremum at $(x_0,y_0)$.

\section*{Second Derivative Test}

Let $z=f(x,y)$ be a function of two variables for which the first and second-order partial derivatives are continuous on some disk containing the point $(x_0,y_0)$. Suppose $f_x(x_0,y_0)=0$ and $f_y(x_0,y_0)=0$.
\[D=f_{xx}(x_0,y_0)f_{yy}(x_0,y_0)-{(f_{xy}(x_0,y_0))}^2\]
\begin{enumerate}
    \item If $D>0$ and $f_{xx}(x_0,y_0)>0$, then $f$ has a local minimum at $(x_0,y_0)$.
    \item If $D>0$ and $f_{xx}(x_0,y_0)<0$, then $f$ has a local maximum at $(x_0,y_0)$.
    \item If $D<0$, then $f$ has a saddle point at $(x_0,y_0)$.
    \item If $D=0$, then the test is inconclusive.
\end{enumerate}

\section*{Absolute Maxima and Minima}

A continuous function $f(x,y)$ on a closed and bounded set $D$ in the plane attains an absolute maximum value at some point of $D$ and an absolute minimum value at some point of $D$.
\vspace{1em}

Assume $z=f(x,y)$ is a differentiable function of two variables defined on a closed and bounded set $D$. Then $f$ will attain the absolute maximum value and the absolute minimum value, which are, respectively, the largest and smallest values found among the following:
\begin{enumerate}
    \item The values of $f$ at the critical points of $f$ in $D$.
    \item The values of $f$ on the boundary of $D$.
\end{enumerate}

A set $E\subseteq\mathbb{R}^n$ is called \textbf{compact} if every sequence of elements ${\{R_k\}}^\infty_{k=1}\subseteq E$ has a convergent subsequence ${\{R_m\}}^\infty_{m=1}$. A subset $E\subseteq\mathbb{R}^n$ is compact if and only if:9
\begin{itemize}
    \item $E$ is closed (containg all its boundary points)
    \item $E$ is bounded ($\exists R > 0, E\subseteq B_R(0)$)
\end{itemize}

\subsection*{Finding Critical Points on Boundary}
Consider a line segment connecting $(a,c)$ and $(b,c)$. It can be parameterized by the equations $x(t)=t$, $y(t)=c$ for $a\leq t\leq b$. Define $g(t)=f(x(t),y(t))$. The critical points are where $g'(t)=0$.

\end{document}