\documentclass{article}
\usepackage{amsmath}
\usepackage{amssymb}
\usepackage[a4paper, top=0.75in, bottom=0.75in]{geometry}

\title{Functions of Several Variables}
\author{David Robinson}
\date{}
\setlength{\parindent}{0pt}

\begin{document}

\maketitle

A \textbf{function of two variables} $z=f(x, y)$ maps each ordered pair $(x, y)$ in a subset $D$ of the real plane $\mathbb{R}^2$ to a unique real number $z$. The set $D$ is called the \textbf{domain} of the function. The \textbf{range} of $f$ is the set of all real numbers $z$ that has at least one ordered pair $(x, y)\in D$ such that $f(x, y)=z$.

\subsection*{Level Curves}

Given a function $f(x, y)$ and a number $c$ in the range of $f$, a \textbf{level curve of a function of two variables} for the value $c$ is defined to be the set of points satisfying the equation $f(x, y)=c$.
\vspace{1em}

Consider a function $z=f(x,y)$ with domain $D\subseteq \mathbb{R}^2$. A \textbf{vertical trace} of the function can be either the set of points that solves the equation $f(a,y)=z$ for a given constant $x=a$ or $f(x,b)=z$ for a given constant $y=b$.

\subsection*{Functions of More Than Two Variables}

Given a function $f(x, y, z)$ and a number $c$ in the range of $f$, a \textbf{level surface of a function of three variables} is defined to be the set of points satisfying the equation $f(x, y, z)=c$.
\end{document}