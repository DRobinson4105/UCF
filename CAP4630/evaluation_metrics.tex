% chktex-file 12
% chktex-file 44

\documentclass{article}
\usepackage{amsmath}
\usepackage{amssymb}
\usepackage{tabularx}
\usepackage{bm}
\usepackage[a4paper, top=0.75in, bottom=0.75in]{geometry}

\title{Evaluation Metrics}
\author{David Robinson}
\date{}
\setlength{\parindent}{0pt}

\begin{document}

\maketitle

\section*{Metrics}

\textbf{Accuracy} measures the number of correct predictions relative to the total predictions.
\[\textbf{Accuracy}=\frac{TP+TN}{P+N}\]
\textbf{Precision} measures how many of the predicted positives are actually positive.
\[\textbf{Precision}=\frac{TP}{TP+FP}\]

\textbf{Recall} measures the ability to identify all relevant instances.
\[\textbf{Recall}=\frac{TP}{TP+FN}\]

\subsection*{Generalized F\bm{$\beta$}-Measure}
$F_\beta$ allows you to adjust the balance between precision and recall, where $\beta > 1$ puts more emphasis on recall and $\beta < 1$ puts more emphasis on precision.
\[F_\beta = (1+\beta^2)\cdot\frac{\text{Precision}\cdot\text{Recall}}{\beta^2\cdot\text{Precision}+\text{Recall}}\]

\textbf{F1-Score} is the harmonic mean of precision and recall, which gives more weight to low values.

\section*{Paired t-Test}

The \textbf{paired t-test} compares two related groups, like scores or results from two different methods applied to the same dataset.
\begin{enumerate}
    \item Calculate the difference between each pair of scores
    \item Calculate the mean of the differences
    \item Find the standard deviation of the differences
    \item Calculate the t-statistic
    \[t=\frac{\bar{d}}{\frac{s_d}{\sqrt{n}}}\]
    \item Compare the t-statistic to a critical value from a t-distribution table and the difference is significant if the t-statistic is larger than the critical value
\end{enumerate}

\end{document}