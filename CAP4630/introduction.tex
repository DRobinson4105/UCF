\documentclass{article}
\usepackage{amsmath}
\usepackage[a4paper, top=0.75in, bottom=0.75in]{geometry}

\title{Introduction}
\author{David Robinson}
\date{}
\setlength{\parindent}{0pt}

\begin{document}

\maketitle

\section*{Supervised Learning}

Supervised learning involves training a model on a labeled dataset, where the correct output is provided for each input.

\begin{enumerate}
    \item \textbf{Data Collection}: Gather labeled data for training
    \item \textbf{Model Training}: Use algorithms to learn the mapping from inputs to outputs
    \item \textbf{Model Evaluation}: Test the model on unseen data to evaluate its performance
\end{enumerate}

\subsection*{Applications}

\begin{enumerate}
    \item \textbf{Linear Regression}: Finds the line of best fit through the data points to predict output based on input features
    \item \textbf{Logistic Regression}: Uses the sigmoid function to map predicted values to probabilities between 0 and 1
    \item \textbf{Decision Trees}: Splits the data into subsets based on the most informative features, forming a tree-like structure
\end{enumerate}

\section*{Unsupervised Learning}

Unsupervised learning involves training a model on data without labeled responses, where the model attempts to identify patterns, relationships, or structures within the data.

\begin{enumerate}
    \item \textbf{Data Collection}: Gather data without labels
    \item \textbf{Model Training}: Use algorithms to find hidden structures within the data
    \item \textbf{Interpretation}: Analyze the output to gain insights or make decisions
\end{enumerate}

\subsection*{Applications}

\begin{enumerate}
    \item \textbf{K-Means Clustering}: Divides the data into K clusters, where each point belongs to the cluster with the nearest mean
    \item \textbf{Principle Component Analysis (PCA)}: Transforms the data into a set of linearly uncorrelated components that capture the most variance
\end{enumerate}

\section*{Reinforcement Learning}

Reinforcement learning involves training a model to learn how to behave in an environment by performing actions and seeing the results.

\subsection*{Applications}

\begin{enumerate}
    \item \textbf{Q-Learning}: Updates a Q-table that maps state-action pairs to expected future rewards
    \item \textbf{Deep Q-Networks (DQN)}: Uses a neural network to approximate the Q-function, allowing it to handle large state spaces
\end{enumerate}

\end{document}