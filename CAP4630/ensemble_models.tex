% chktex-file 12
% chktex-file 44

\documentclass{article}
\usepackage{amsmath}
\usepackage{amssymb}
\usepackage{tabularx}
\usepackage{bm}
\usepackage{graphicx}
\usepackage[a4paper, top=0.75in, bottom=0.75in]{geometry}

\graphicspath{{./images/}}

\title{Ensemble Models}
\author{David Robinson}
\date{}
\setlength{\parindent}{0pt}

\begin{document}

\maketitle

\section*{Ensemble Models}

\textbf{Ensemble modeling} is a technique in machine learning that combines multiple models to achieve better predictive performance.

\subsection*{Common Ensemble Techniques}
\begin{itemize}
    \item \textbf{Bagging} reduces variance by training models on different random samples and averaging their predictions, such as random forest.
    \item \textbf{Boosting} reduces bias by sequentially building models that correct errors made by the previous one, such as AdaBoost and Gradient Boosting.
    \item \textbf{Stacking} combines preditions from different types of strong learners by training a meta-model on their outputs to improve final prediction accuracy.
\end{itemize}

\subsection*{Expected Test Error}
\[\mathbb{E}_{D\sim P_n(x,y)\sim P}[{(f_D(x)-y)}^2]=\text{Variance}+\text{Bias}+\text{Noise}\] where
\begin{itemize}
    \item \textbf{Variance} $=\mathbb{E}_{x,D}[{(f_D(x)-\bar{f}(x))}^2]$ measures the variability of the predictions from model trained on subset $D$, $f_D(x)$, around the average prediction $\bar{f}(X)$.
    \item \textbf{Bias} $=\mathbb{E}_x [{(\bar{f}(x)-\bar{y}(x))}^2]$ measures the difference between the average model prediction $\bar{f}(x)$ and the true value $\bar{y}(x)$.
    \item \textbf{Noise} $=\mathbb{E}_{x,y} [{(\bar{y}(x)-y)}^2]$ represents the randomness in the data.
\end{itemize}

\subsection*{Random Forest}
Random Forest is a bagging-based ensemble method.
\begin{enumerate}
    \item Draw $m$ samples from the original dataset $D$.
    \item Train an independent decision tree for each sample.
    \item At each node split within a tree, randomly select a subset of $k\leq d$ features, where $d$ is the total number of features, and choose the best split only from this subset
\end{enumerate}

\end{document}